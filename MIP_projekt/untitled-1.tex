\documentclass{article}
\usepackage[utf8]{inputenc}
\usepackage[slovak]{babel}


\begin{document}
\title{Agilné metódy vs tradičné vývojové metódy}
\author{Juraj Majerník}
\date{21. Oktobra 2021}
\maketitle
\begin{abstract}
%abstrakt zo zošita tu
\end{abstract}
\section{Úvod}
Témou práce sú agilné metódy vs tradičné vývojové metódy v softvérovom inžinierstve. V dnešnej dobe je táto problematika veľmi konzultovanou témou. 
V našej práci chceme zistiť dôvod, pre ktorý ešte stále nenahradili agilné metódy tradičné metódy, keďže sú efektívnejšie a flexibilnejšie. Flexibilita je zároveň v dnešnej dobe vyhľadávaná schopnosť, keďže nám dovoľuje sa prispôsobiť zmenám od klientov a tým nám pomáha šetriť čas a peniaze pri dodaní nášho riešenia. 
V prvej časti článku sa venujem agilnému vývoju ako takému. Dozvieme sa tu, čo to agilný vývoj vlastne je, jeho výhody a nevýhody, prečo boli vyvinuté a načo sa používajú. 
V druhej časti článku sa venujem extrémnemu programovaniu. Dozvedáme sa tu o jeho problematike, jeho výhody a nevýhody. 
V tretej časti článku si priblížime problematiku vodopádového vývoja. 
Vo štvrtej časti článku si povieme o iteratívnom a inkrementálnom vývoji, ich plusoch a mínusoch a problematike ako takej. 
V piatej časti článku sa dozvieme niečo o RAD vývoji. 
V poslednej – šiestej časti článku sa venujeme rozdielom medzi jednotlivými metódami a porovnávame ich výhody a nevýhody. 
Cieľom práce je zistiť či sú agilné metódy vhodné na nahradenie všetkých zvyšných procesov. Predstavíme si tieto základné techniky, popíšeme si ich základné informácie, vymenujeme výhody a nevýhody z pohľadu na flexibilitu, časovej náročnosti a robustnosti. V každej časti článku poukážeme na prípady softvéru, pre ktoré sú jednotlivé metódy vhodné a nevhodné a oddôvodnime si ich. V závere odpovieme na otázku či môžu nahradiť agilné metódy tradičné vývojové metódy.

\section{Agilné vývojové metódy}

\section{Extrémne programovanie}

\section{Vodopádový vývoj}

\section{Iteratívny a Inkrementálny vývoj}

\section{Špirálový vývoj}

\section{RAD vývoj}

\section{Porovnanie výhod a nevýhod}

\section{Záver}





\end{document}